\documentclass[a4paper,10pt, notitlepage]{report}
\usepackage[utf8]{inputenc}
\usepackage{natbib}
\usepackage{amssymb}
\usepackage{amsmath}
\usepackage{enumitem}
\usepackage{xcolor}
\usepackage{url}
\usepackage{cancel}
\usepackage{mathtools}
\usepackage[portuguese]{babel}
\usepackage{newclude}

%%%%%%%%%%%%%%%%%%%% Notation stuff
\newcommand{\pr}{\operatorname{Pr}} %% probability
\newcommand{\vr}{\operatorname{Var}} %% variance
\newcommand{\rs}{X_1, X_2, \ldots, X_n} %%  random sample
\newcommand{\irs}{X_1, X_2, \ldots} %% infinite random sample
\newcommand{\rsd}{x_1, x_2, \ldots, x_n} %%  random sample, realised
\newcommand{\bX}{\boldsymbol{X}} %%  random sample, contracted form (bold)
\newcommand{\bx}{\boldsymbol{x}} %%  random sample, realised, contracted form (bold)
\newcommand{\bT}{\boldsymbol{T}} %%  Statistic, vector form (bold)
\newcommand{\bt}{\boldsymbol{t}} %%  Statistic, realised, vector form (bold)
\newcommand{\emv}{\hat{\theta}}
\DeclarePairedDelimiter\ceil{\lceil}{\rceil}
\DeclarePairedDelimiter\floor{\lfloor}{\rfloor}
\DeclareMathOperator*{\argmax}{arg\,max}
\DeclareMathOperator*{\argmin}{arg\,min}
%%%%
\newif\ifanswers
\answerstrue % comment out to hide answers

% Title Page
\title{Primeira avaliação (A1)}
\author{Disciplina: Inferência Estatística \\ Instrutor: Luiz Max Carvalho \\ Monitores: Jairon Nóia \& Tiago Silva}
\date{24 de Setembro de 2022}

\begin{document}
\maketitle

\begin{center}
\fbox{\fbox{\parbox{1.0\textwidth}{\textsf{
    \begin{itemize}
        \item O tempo para realização da prova é de 3 horas;
        \item Leia a prova toda com calma antes de começar a responder;
        \item Responda todas as questões sucintamente;
        \item Marque a resposta final claramente com um quadrado, círculo ou figura geométrica de sua preferência;
        \item A prova vale 80 pontos. A pontuação restante é contada como bônus;
        \item Apenas tente resolver a questão bônus quando tiver resolvido todo o resto;
        \item Você tem direito a trazer \textbf{uma folha de ``cola''} tamanho A4 frente e verso, que deverá ser entregue junto com as respostas da prova.
    \end{itemize}}
}}}
\end{center}

\newpage

\section*{1. Treasure map.}

Suponha que temos um modelo estatístico paramétrico, com f.d.p./f.m.p. $f_\theta(x)$, $\theta \in \Omega \subseteq \mathbb{R}^p$ com suporte em $\mathcal{X} \subseteq \mathbb{R}^d$.
Dada uma observação $\bX=\bx$, o chamado estimador \textit{maximum a posteriori}, MAP, é definido como 
\begin{equation*}
    \delta_{\text{MAP}}(\bX) = \argmax_{\theta \in \Omega} \xi(\theta \mid \bx).
\end{equation*}

\begin{enumerate}[label=\alph*)]
 \item (10 pontos) Mostre que quando $\Omega = \{\theta_1, \theta_2, \ldots, \theta_k\}$, $k\geq2$, isto é, quando o espaço de parâmetros é discreto, $\delta_{\text{MAP}}$ é o estimador de Bayes sob  a seguinte perda
 \begin{equation*}
     L(\delta, \theta) = 
     \begin{cases}
     0, \delta = \theta,\\
     1, \text{caso contrário},
     \end{cases}
  \end{equation*}
  também chamada de perda \textit{zero-um} (\textit{0-1} loss).    
 \item (10 pontos) Suponha que a proporção $\theta$ de itens defeituosos em uma linha de produção toma apenas os valores $0,1$ e $0,2$.
 Suponha ainda que $n$ itens são inspecionados  e $x$ são defeituosos, $x \in \{0, 1, \ldots, n\}$.
 Mostre como encontrar o estimador de máxima verossimilhança para $\theta$;
 \item (10 pontos) Suponha que, \textit{a priori}, $\pr(\theta = 0,1) =: \pi(0,1) = 0,7$. 
 Exiba a distribuição \textit{a posteriori} de $\theta$ e mostre como encontrar o MAP para $\theta$.
 \end{enumerate}
\ifanswers
\include*{A1_2022_sol1}
\fi

\section*{2. Now, Dinah, tell me the truth.} 

Tome $\bx = (x_{1}, \dots, x_{n}) \in \mathbb{N}^{n}$ um conjunto de realizações de uma variável aleatória $X$ distribuída conforme a distribuição de Poisson, com f.m.p.
\begin{equation*} 
	f(x | \theta) = \frac{\theta^{x}}{x!} e^{-\theta}, 
\end{equation*} 
\noindent com taxa $\theta \in \mathbb{R}_{+}$ desconhecida. 

\begin{enumerate}[label=\alph*)] 
	\item (10 pontos) Defina $\bar{X}_{n} := \frac{1}{n} \sum_{i=1}^n X_{i}$.
 Verifique que os estimadores 
		\begin{equation*} 
			\delta_{1}(\mathbf{X}) = \frac{1}{n} \sum_{i=1}^n X_{i} \text{ e } \delta_{2}(\mathbf{X}) = \frac{1}{n - 1}  \sum_{i=1}^n (X_{i} - \bar{X}_{n})^{2},  
		\end{equation*} 
		\noindent são não viesados de $\theta$. 
	\item (10 pontos)  Mostre que $\delta_{1}$ é eficiente.
 Ele é consistente? 
	\item (10 pontos)  Suponha que $n = 2$.
 Mostre que $\delta_{2}$ é \textbf{inadmissível}.
 
 \textbf{Dica}: Se $X$ e $Y$ são variáveis aleatórias com distribuição de Poisson com média $\theta$, então 
	\begin{equation*}
	    E_\theta\left[\frac{(X - Y)^{4}}{4}\right] = 3\theta^{2} + \frac{\theta}{2}. 
	\end{equation*}
 	\item (10 pontos) A informação de Fisher quantifica a informação sobre um parâmetro contida em uma amostra aleatória.
 Compute a informação de Fisher, $I_{X}(\theta)$, em $\mathbf{X}$. 
	\item (10 pontos)  A parametrização é crucial para a informação de Fisher; existem transformações de variáveis que mudam substancialmente a sua interpretação.
 Sendo assim, prove que para o modelo Poisson, a informação de Fisher em $\mathbf{X}$ sobre $\eta = \sqrt{\theta}$ é constante.
 
 \textbf{Dica}: Se $\eta = g(\theta)$, então $I_{X}(\theta) = I_{X}(\eta) |g'(\theta)|^{2}$. 
			% $g \colon \mathbb{R}_{+} \rightarrow \mathbb{R}_{+}$
\end{enumerate} 
\ifanswers
\include*{A1_2022_sol2}
\fi

\section*{3. \textit{Actually}, Beta wolves don't exist in the wild\footnote{\url{https://sciencenorway.no/ulv/wolf-packs-dont-actually-have-alpha-males-and-alpha-females-the-idea-is-based-on-a-misunderstanding/1850514}}.}

Em várias aplicações estatísticas os dados nos são apresentados na forma de proporções.
Um bom exemplo é a proporção de óleo bruto que é convertida em gasolina depois da destilação e fracionamento.
Para modelar estes dados é preciso escolher um modelo apropriado.
A distribuição Beta é uma família de distribuições contínuas com suporte em $(0, 1)$, cuja densidade (no suporte) vale
\begin{equation*}
f(x; a, b) = \frac{\Gamma(a + b)}{\Gamma(a)\Gamma(b)} x^{a-1} (1-x)^{b-1},
\end{equation*}
para $a, b > 0$.
Para esta distribuição, sabemos que $E_\theta[X] = a/(a + b)$ e $\vr_\theta(X) = ab/[(a+b)^2(a+b+1)]$.
Tome $\bX = (\rs)$ uma amostra aleatória de uma distribuição Beta com parâmetros $a$ e $b$.

\begin{enumerate}[label=\alph*)]
 \item (10 pontos) Encontre uma estatística suficiente para $a$ quando $b$ é conhecida.
 \item (10 pontos) Encontre o estimador de máxima verossimilhança para $E_\theta[X]$ quando $b = 1$, conhecido.
 \end{enumerate}
\ifanswers
\include*{A1_2022_sol3}
\fi

\section*{Bônus: Lindex!}

A função de perda LINEX (LINear--EXponential) é uma função de perda que trata assimetrias de maneira suave.
Essa função é definida como:
\[ L(\theta, a) = e^{c(a-\theta)} - c(a-\theta) -1, \]
onde $c>0$.
Quando $c$ varia, a função de perda varia de muito assimétrica para quase simétrica.
\begin{enumerate}[label=\alph*)]
\item (10 pontos) Mostre que o estimador de Bayes para $\theta$ é dado por 
\begin{equation*}
    \delta(\bX) = -\frac{1}{c} \log \left( E\left[e^{-c\theta} \, | \, \bX) \right] \right).
\end{equation*}
\item (10 pontos) Seja $X_1,\ldots,X_n$ uma amostra aleatória de uma distribuição $N(\mu,\sigma^2)$, com $\sigma^2$ conhecido.
Suponha ainda que a priori é não informativa, ou seja $p(\mu) \propto 1$.
Mostre que o estimador de Bayes utilizando a perda LINEX é 
\[\widehat{\theta}_L = \overline{X}_n - \frac{c \sigma^2}{2n}.\]
\end{enumerate}
\textbf{Dica:} Se $Z$ é uma variável aleatória com distribuição normal de média $m$ e variância $v$,
\begin{equation*}
    E[\exp\{kZ\}] = \exp\left(\frac{k^2v + 2km}{2}\right),
\end{equation*}
para $k \in \mathbb{R}$.
\ifanswers
\include*{A1_2022_sol_bonus}
\fi

% \bibliographystyle{apalike}
% \bibliography{refs}

\end{document}          
