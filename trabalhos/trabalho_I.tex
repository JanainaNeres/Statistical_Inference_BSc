\documentclass[a4paper,10pt, notitlepage]{report}
\usepackage[utf8]{inputenc}
\usepackage{natbib}
\usepackage{amssymb}
\usepackage{amsmath}
\usepackage{enumitem}
\usepackage[portuguese]{babel}


% Title Page
\title{Trabalho I: Análise bayesiana no caso Normal.}
\author{Disciplina: Inferência Estatística \\ Professor: Luiz Max de Carvalho}

\begin{document}
\maketitle

\textbf{Data de Entrega: 22 de Agosto de 2022.}

\section*{Orientações}
\begin{itemize}
 \item Enuncie e prove (ou indique onde se pode encontrar a demonstração) de~\underline{todos} os resultados não triviais necessários aos argumentos apresentados;
 \item Lembre-se de adicionar corretamente as referências bibliográficas que utilizar e referenciá-las no texto;
 \item Equações e outras expressões matemáticas também recebem pontuação;
 \item Você pode utilizar figuras, tabelas e diagramas para melhor ilustrar suas respostas;
 \item Indique com precisão os números de versão para quaisquer software ou linguagem de programação que venha a utilizar para responder às questões\footnote{Não precisa detalhar o que foi usado para preparar o documento com a respostas. Recomendo a utilização do ambiente LaTeX, mas fique à vontade para utilizar outras ferramentas.};
 \end{itemize}


\section*{Introdução}

A distribuição Normal (ou gaussiana) é largamente utilizada na prática estatística, por uma série de razões matemáticas e históricas~\citep{Kim2008}.
No campo da estatística bayesiana, algumas manipulações simples permitem a análise do caso gaussiano em forma fechada.
Neste trabalho, vamos derivar os principais resultados de uma análise bayesiana conjugada de dados normalmente distribuídos.
Para tal, começamos com uma reparametrização.
Em particular, fazemos $\tau = 1/\sigma^2$, de modo que os parâmetros de interesse $\theta = (\mu, \sigma^2)$ se tornem $\phi = (\mu, \tau)$.
O parâmetro $\tau$ é chamado de~\textit{precisão}.
Suponha que observamos uma amostra aleatória $X_1, \ldots, X_n$ com distribuição normal com parâmetros $\mu$ e $\tau$, ambos desconhecidos.

\section*{Questões}
\begin{enumerate}
 \item Escreva a distribuição conjunta condicional dos dados sob a nova parametrização;
 \item A partir da densidade do item anterior, deduza que a distribuição~\textit{a priori} conjugada conjunta para $\phi = (\mu, \tau)$ é da forma:
 \begin{align}
  \tau &\sim \operatorname{Gama}(\alpha_0, \beta_0),\\
  \mu \mid \tau &\sim \operatorname{Normal_2}(m_0, \lambda_0\tau),
 \end{align}
onde $\operatorname{Normal_2}$ se refere à distribuição normal parametrizada em termos de média e precisão.
\item A partir dos itens anteriores, derive a distribuição~\textit{a posteriori} conjunta de $\mu$ e $\tau$ e a distribuição condicional de $\mu$ dado $\tau$, assim como a distribuição marginal~\textit{a posteriori} de $\tau$;
\item Interprete as expressões obtidas no item anterior; o que as formas funcionais obtidas revelam sobre a interação entre os hiperparâmetros e os dados?
\item Derive a distribuição marginal~\textit{a posteriori} de $\mu$ (Dica: leia o capítulo 8.4 de De Groot);
\item Palmirinha anda preocupada com a concentração de amido em sua pamonha.
Ela pede para Valciclei, seu assistente, amostrar $n=10$ pamonhas e medir sua concentração de amido.

Ele, muito prestativo, rapidamente faz o experimento, mas, porque comeu todas as amostras depois que foram medidas, precisou fazer uma visita de emergência ao banheiro. 
Desta feita, apenas teve tempo de anotar em um papel a média e variância amostrais, $\bar{x}_n =  8.307849$ e $\bar{s}^2_n = 7.930452$.

Palmirinha tem uma reunião com investidores em pouco tempo, então decide voltar aos seus tempos de bayesiana~\textit{old school} e analisar os dados utilizando prioris conjugadas.
Ela supõe que a concentração de amido segue uma distribuição normal com parâmetros $\mu$ e $\tau$ e que as observações feitas por Valciclei são independentes entre si.
Ela suspeita que a concentração de amido na pamonha fique em torno de $10$ mg/L, com desvio padrão de  $2$ mg/L.
Com sua larga experiência na confecção de pamonhas, ela suspeita ainda que o coeficiente de variação da concentração de amido seja em torno de $1/2$.
Palmirinha tem um quadro em seu escritório, que diz
\[ \operatorname{cv} = \frac{\sigma}{\mu}. \]

Agora, 
\begin{enumerate}
 \item Com os dados anotados por Valciclei, é possível computar a distribuição~\textit{a posteriori} de $\mu$ e $\tau$? Justifique.
 \item Em caso afirmativo, ajude Palmirinha a encontrar $a, b \in \mathbb{R}$, $a < b$ de modo que $\operatorname{Pr}(\mu \in (a, b) \mid \boldsymbol{x}) = 0.95$.
\end{enumerate}

    
\end{enumerate}



\bibliographystyle{apalike}
\bibliography{refs}

\end{document}          
